\chapter{Related Work}
\label{chapter:Related Work}


\section{Literature}
\subsection{Vision-based Segmentation/Recognition}

%Traditionally, research in passive perception was concentrated on finding a set of features (e.g. color~\cite{Balch_2000_3584}) or higher order features (e.g. those existing in graph cut approaches~\cite{BoykovGraphCut}) in static and segmented images.

%Segmentation/Identitation of new objects - ICRA 2013


%General recognition system (TangMillerSingh) - combining 3d data and sift

%Object search using high level information -ICRA2013preprint

%Object recognition combining rgb and d - kernel descriptors - icra 11b

%new features combining rgb and d - drost20123

%bag of words approaches cciw2013

%labeling objects using online datasets - 3d objects labeling ICRA

%Datasets - K. Lai, L. Bo, X. Ren, and D. Fox. A large-scale hierarchical multi-
%view RGB-D object dataset. In Proc. of the IEEE International Conference on Robotics and Automation (ICRA), 2011. CALTECH


%textureless object - LINEMOD method - Hinterstoisser and improved

%transparent objects - p35

In this part of the thesis we look at the big picture of research that has been done in the areas of static object segmentation and static object recognition. 

One of the most successful approaches in static segmentation has been shown by Marton et al.~\cite{marton12SC} where the authors presented an algorithm that makes use of over-segmentation of objects and part-graph-based hashing. In our work we use their algorithm as a prior for the interactive segmentation pipeline (it is described in detail in Chapter \ref{chapter:Textureless Segmentation}). Zhan et al.~\cite{RGBRegionGrowing} described a different approach where the emphasis is put on colorimetrical and spatial relations between the extracted regions of a point cloud. The regions are first formed using a kd-tree based region growing algorithm that searches k nearest neighbors of a given point. The last step of the algorithm merges and refines the regions obtained in the first step. There have been also a number of graph-based algorithms that approach the object segmentation problem. Felzenschwalb and Huttenlocher~\cite{Felzenszwalb} presented a graph-based algorithm revolving around a predicate that measures the likelihood of a boundary between two regions.
The results of all three static segmentation algorithms described above are shown in Figure \ref{fig:tracking_dists}.  

Another example of a graph-based image segmentation algorithm was presented in work by Boykov and Kolmogorov~\cite{BoykovGraphCut} where the authors evaluate different min-cut/max-flow algorithms in order to approximate energy minimization in low-level vision. 

A different approach to the image segmentation problem was presented by Ouadfel and Batouche~\cite{ouadfel2003mrf}. The authors developed an algorithm that is based on an ant colony system. The main idea behind the ant colony approach is that in order to improve image regions labelling, ant-inspired communication techniques between local agents are leveraged. Cooperation between ants uses the idea of pheromone updates where ants leave their trace in order to enable other ants to follow them.  
 
Similarly to the topic of static object segmentation there has been a vast amount of research in the area of object recognition. 

One of the most efficient and robust object recognition systems was developed by Tang et al.~\cite{tang2012textured}. It is worth mentioning that the system took the 1st place in the ICRA 2011 Solutions in Perception instance recognition challenge. The system can be divided into several steps. In the training phase a full object model is created, then it is transformed into a mesh, and finally, the RGB data is projected onto the object model acquired in the previous steps. The authors decided to use SIFT features (described in detail in Chapter \ref{chapter:Background}) and color histograms in order to fully describe the textural property of the object. In the testing phase plane extraction and euclidean clustering algorithms are employed in order to segment the object. As next steps color model matching and SIFT feature matching are used to recognize the object. The algorithm also performs geometric pose estimation as well as final scene verification and refinement using global scene consistency checks. 

A different approach was discussed by Weijer and Khan~\cite{van2013fusing} where the authors compare various bag-of-words based recognition algorithms. The algorithm represents an image as a set of local regions where each of them is represented as a visual vocabulary. Different objects correspond to different histograms (called bags-of-words) over the created vocabulary. An extracted bag-of-words histogram can be compared to all the histograms stored in the memory and thus, an object can be labelled as one of the previously seen objects. 

There has also been plethora of research that concentrates on the shape of objects rather than their texture. 

Such an approach was presented by Teo et al.~\cite{teoembedding} where the authors tackle the problem of object search which has a similar formulation to the object recognition problem. The algorithm uses a modified version of the image torque~\cite{nishigaki2012image} which is augmented by using high level knowledge of the target object's properties. The input of the torque operator contains the visible edges and it calculates how close they are to form a closed, convex contour.

Drost and Ilic~\cite{drost20123d} presented an object detection approach based on surface descriptors. The authors use geometric edges (extracted from the intensity image and filtered using the range image) in order to build a rotation and scale invariant surface descriptor. The evaluation showed that the system is particularly robust with respect to occlusions.  

There is also a lot of research on other categories of objects that were mentioned in Chapter \ref{chapter:Introduction}: a system that is able to robustly detect textureless objects presented by Hinterstoisser et al.~\cite{hinterstoisser2011multimodal},and an algorithm that enables robots to recognize and estimate the pose of transparent objects showed by Lysenkov et al.~\cite{lysenkov2012recognition}, to name but a few. 

\newpage

\subsection{Interactive Perception}



In this thesis we take a different approach to the ones presented at the beginning of this section. We create an interactive scene segmentation and by including robotic arm manipulation into the perception loop. Fitzpatrick~\cite{fitzpatrick_active_vision} and Kenney et al.~\cite{KenneyInteractive} discussed the problem of segmenting rigid objects from a video stream. These video streams presented objects being moved by the robot (pre-planned arm motion) and used a simple Gaussian model of the color values to deduce a possible motion and the graph cut algorithm for the final object segmentation. Their methods are accurate for textured and textureless objects. The difference between their and our approach is that we did not use the pre-planned arm motion. Instead, our arm motion adapts to the scene and enables us to use the 3D data to segment the object aspirants from the background.

The problem of segmenting articulated objects was discussed by Katz et al.~\cite{Katz-WS-MM-ICRA2011}. In this approach, a Lucas-Kanade tracker as well as a set of predictors (short and long distance, color, relative motion, triangulation and fundamental matrix) were used in order to achieve rigid body hypotheses, which took form of a graph. Additionally, they obtained subsequent fixation point on the object which enables segmenting an object based on intensity, color and texture cues. There are some important limitations in this approach. First, the pre-planned arm motion and second, the time needed to break the object hypotheses into subgraphs by using a min-cut algorithm. 
Another approach to interactive segmentation is presented by Bergstrom et al.~\cite{bergstrom11icvs}. Their methodology requires preliminary labelling using a 3D segmentation through fixation whose outcome is a rough initial segmentation. Here, the robot makes an interaction with the scene aiming to disambiguate the hypotheses. Consequently, points in motion space are clustered by using a two component Gaussian mixture model. In this case, the limitation is that the number of objects per scene will never be greater than two.
The approaches described above are relevant to the textured objects only. 

There are some approaches that try to address the problem of planning the perturbations in order to accumulate a sequence of motion cues. Cheng et al.~\cite{chang11interactive} present one of them. In their research they developed a framework for interactive segmentation of individual objects, adding an interaction strategy. It enables scene state update, an iterative object selection, evaluation and manipulation primitive selection. The latter uses a set of heuristic that maximize the push action. Nevertheless, the contribution of this component to the successful segmentation of the objects is still unclear. The manipulation primitive evaluation step makes use of sparse correspondences from the Lukas-Kanade optical flow tracker. Then, it computes a set of transforms which match against a dense point cloud colorwise. The magnitude of the transform motion and the percentage of dense point matches determine the likelihood of a target being a single item or multiple items. The limitation of this framework, in comparison to our work, is lack of estimation of the corner contact points.
Gupta et al.~\cite{gupta11primitives}, on the other hand, present the idea of using a set of motion primitives that consist of pick and place, spread and tumble actions in order to sort cluttered piles of single-color objects. They use Euclidean clustering in the distance and the color space to categorize the scenes as uncluttered, cluttered or piled. Both distance and color space have their limitations. Distance-based clustering depends on a correctly selected threshold and color-based clustering relies on stable lighting conditions. This system also presumes that the size of the objects (LEGO bricks) is similar. 

There is a plethora of research that focuses on the estimation of the articulation models for boxes, drawers and similar objects~\cite{Box11Saxena,
  sturm10rss-workshop}. It is usually the case that these approaches assume the existence of a large, moving plane that can be reliably detected by using e.g. a RANSAC algorithm on the input point cloud and that  unanimously represents part of the objects that is searched.


\subsection{Interactive Segmentation for Textured Objects}
In my previous master thesis conducted at Warsaw University of Technology~\cite{polishthesis} I was working on the problem of interactive segmentation algorithm for textured objects. In addition, the preliminary work on interactive segmentation for textureless objects was introduced. The main differences in this thesis are: 

\begin{itemize} 
\item a thorough evaluation of the interactive segmentation of textureless objects system conducted on multiple scenes, 
\item additional algorithms such as dense model reconstruction, and substantial improvements to the ones presented before, 
\item validation of our push heuristic proven by experiments in a simulator,
\item preliminary work on the interactive object recognition method, 
\item introduction of Interactive Perception Library and the concept of the Interactive Perception community. 
\end{itemize} 



%In my previous master thesis conducted at ... I was working on ... The difference in this work is a b and c.
%
%In contrast to our previous work, we investigate here ...  " "Our previous work did not include a,b,c ,...
%
%In our previous work~\cite{polishthesis,bersch12interactive} we developed an interactive segmentation algorithm for textured objects. The idea behind this system is similar to the interactive segmentation system for textureless objects described in this thesis.
%

%
%In our previous work~\cite{bersch12interactive} we focused on an interactive segmentation system for textured objects. 

The idea behind the system for textured objects~\cite{bersch12interactive} is similar to the interactive segmentation system for textureless objects described in this thesis.
Firstly, we extract 2D features which are then tracked while the robot actively induces
motions into a scene using its arm. Then we employ a novel RANSAC-based clustering algorithm in order to assign tracked feature trajectories to the corresponding objects. The number of objects is uknown a priori. The approach was also evaluated on the real robot in a number of scenes.
%
%In those works we also introduced our initial idea of the system that is able to segment objects without texture. This is continued as the main subject (together with additional contributions such as Interactive Object Recognition and Interactive Perception Library) of this thesis.  

