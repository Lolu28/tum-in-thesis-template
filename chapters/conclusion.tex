\chapter{Conclusion}
\label{chapter:conclusion}

\section{Summary}

In this thesis we presented an interactive segmentation system that works with textureless objects of preferably box-like or cylindrical shapes. The system consists of multiple steps that involve algorithms such as initial classification, feature extraction and tracking, push point estimation, trajectory clustering and the dense model reconstruction. The whole system was thoroughly evaluated on multiple scenes and showed significant improved compared to static segmentation techniques.

In the second part of the thesis we showed the initial work on object recognition system that leveraged the same idea of robot interacting with its surrounding. The implemented system is based on feature matching algorithm. Our evaluation proves that bringing an object to its original pose significantly improves the probability of the object being recognized correctly. We also show theoretical background for the rotational push that can be easily implemented on the robot while still remaining efficient in bringing the object into its original pose. 

Finally, we presented the interactive perception software framework that enables to combine the community effort in the interactive perception field. Generic and modular infrastructure makes it possible to run different algorithms within the same system pipeline that it similar in most of the recent interactive perception systems.

All together this thesis shows a shift towards the idea of including the manipulation in the perception loop and it shows promising results while following this approach. 


\section{Future Work}


