\clearemptydoublepage
\phantomsection
\addcontentsline{toc}{chapter}{Abstract}

\vspace*{1cm}
\begin{center}
{\Large \bf Abstract}
\end{center}
\vspace{1cm}

Perception remains one of the most crucial yet still unsolved challenges in robotics today. As robotic platforms become affordable, there appear new opportunities to improve known perception algorithms by making use of robot's capabilities. In this thesis, we present interactive perception methods that leverage the idea of a robot interacting with its surroundings in order to improve its perception.  We first show an interactive segmentation algorithm that works with textureless objects of preferably box-like or cylindrical shapes. In the second part of the thesis we show preliminary work on an interactive object recognition system. Finally, we present an interactive perception software framework that enables researches to combine the community efforts in the interactive perception field. As a result, this thesis shows a shift towards the idea of including manipulation in the perception loop and it presents promising outcomes while following this approach.

%  The ability to understand data acquired from a visual sensor has been a goal that  many communities tried to tackle in the last 30 years. 

%In this thesis, we first present an interactive segmentation system that works with textureless objects of preferably box-like or cylindrical shapes. The system consists of multiple steps that involve various algorithms such as initial classification, feature extraction and tracking, push point estimation, trajectory clustering, and the dense model reconstruction. The whole system was thoroughly evaluated on multiple scenes and showed significant improvement compared to static segmentation techniques.

%In the second part of the thesis we show preliminary work on an object recognition system that also leverages the idea of a robot interacting with its surroundings in order to improve its perception. The implemented system is based on the feature matching algorithm. Our evaluation proves that bringing an object to the pose stored in the database significantly improves the probability of the object being recognized correctly. We also show theoretical background for the rotational push that can be easily implemented on the robot while still remaining efficient in bringing the object into its original pose. 

%Finally, we present an interactive perception software framework that enables researches to combine the community effort in the interactive perception field. Generic and modular infrastructure makes it possible to run different algorithms within the same system pipeline, which is similar in most of the recent interactive perception systems.

%In summary, this thesis shows a shift towards the idea of including manipulation in the perception loop and it presents promising results while following this approach.