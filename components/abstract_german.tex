\selectlanguage{german}

\clearemptydoublepage
\phantomsection
\addcontentsline{toc}{chapter}{Zusammenfassung}

\vspace*{1cm}
\begin{center}
{\Large \bf Zusammenfassung}
\end{center}
\vspace{1cm}



Perzeption verbleibt bis zum heutigen Zeitpunkt eines der entscheidendsten und trotzdem ungel{\"o}sten Probleme der Robotik.
Da Roboterplattformen immer erschwinglicher werden, ergeben sich neue M{\"o}glichkeiten State-Of-The-Art Perzeptionsalgorithmen
durch den Einsatz neuer Robotertechnologie zu verbessern.
In dieser Masterarbeit zeigen wir, wie durch interaktive Perzeptionstechniken, bei denen ein Roboter mit seiner Umgebung interagiert,
Perzeption verbessert werden kann.
Zuerst stellen wir einen interaktiven Segmentierungsalgorithmus vor, der mit untexturierten,
haupts{\"a}chlich kubischen oder zylindrischen Objekten arbeitet.
Des Weiteren zeigen wir eine vorl{\"a}ufige Version eines interaktiven Objekterkennungssystems.
Schlussendlich stellen wir ein interaktives Softwareframework f{\"u}r Perzeption vor,
das es Forschern erm{\"o}glicht Ergebnisse und Fortschritte in diesem Bereich leichter auszutauschen.
Zusammenfassend beschreiben wir in dieser Thesis die Herangehensweise, Manipulation und
Perzeption zu vereinen und pr{\"a}sentieren vielversprechende Ergebnisse.


\selectlanguage{english}
